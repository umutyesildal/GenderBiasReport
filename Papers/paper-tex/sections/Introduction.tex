The rapid adoption of Large Language Models (LLMs) in educational technology has revolutionized content creation, offering unprecedented capabilities for generating instructional materials, assessments, and learning resources \cite{openai2023gpt4}. However, these powerful systems inherit and amplify gender biases present in their training data, potentially perpetuating stereotypes and inequitable representations in educational contexts \cite{urchs2024chatgptbias, zhao2024multilangbias}.

Gender bias in educational content poses significant pedagogical and ethical concerns. Biased language can reinforce stereotypes, limit students' aspirational horizons, and create exclusionary learning environments that particularly disadvantage underrepresented groups \cite{you2024beyondbinary}. As educational institutions increasingly integrate AI-generated content into curricula, ensuring gender neutrality becomes not merely a technical challenge but a fundamental requirement for inclusive education.

Recent research has demonstrated that structured prompting strategies can effectively mitigate gender bias in LLM outputs \cite{zeng2024debiasprompting, savoldi2024neutraltranslation}. However, existing studies often focus on general text generation or specific linguistic tasks, leaving a critical gap in understanding how different prompting approaches perform specifically within educational content domains where clarity, accuracy, and pedagogical effectiveness must be balanced with inclusivity.

\subsection{Research Objectives and Contributions}

This paper addresses the following research question: \textit{Which prompting strategy most effectively reduces gender bias in LLM-generated educational content while maintaining text quality and semantic fidelity?}

Our study makes several key contributions:

\begin{enumerate}
    \item \textbf{Comprehensive Experimental Framework}: We present a systematic evaluation of four distinct prompting strategies across multiple LLM architectures, providing robust empirical evidence for bias mitigation effectiveness.
    
    \item \textbf{Educational Domain Focus}: Unlike previous studies that examine general text generation, our research specifically targets educational content, ensuring practical relevance for pedagogical applications.
    
    \item \textbf{Multi-dimensional Evaluation}: We introduce a comprehensive evaluation framework encompassing gender bias reduction, text fluency, and semantic preservation, addressing the critical trade-offs inherent in bias mitigation.
    
    \item \textbf{Practical Guidelines}: Based on our findings, we provide evidence-based recommendations for educators and educational technology developers seeking to implement gender-neutral content generation systems.
\end{enumerate}

\subsection{Paper Organization}

The remainder of this paper is organized as follows: Section II reviews related work on gender bias in LLMs and prompting strategies. Section III details our experimental methodology, including corpus construction, prompting strategies, and evaluation metrics. Section IV presents our results from 300 controlled experiments. Section V discusses implications and limitations, while Section VI concludes with recommendations for future research and practical implementation.