
\subsection{Understanding Gender Bias}

Gender bias in language refers to systematic and unfair preferences towards a particular gender, reflected through language usage, roles, and stereotypes. Recently, the presence of gender bias within large language models (LLMs) has become an important issue, given their wide-ranging application in educational, professional, and social settings. Such biases, if left unaddressed, can perpetuate stereotypes, inequalities, and reinforce discriminatory practices.

In educational contexts specifically, gender neutrality is crucial. Gender-biased language can negatively affect students' identities, perpetuate stereotypes, and hinder the establishment of inclusive learning environments. Ensuring neutrality in educational resources promotes equity, respect, and inclusion.

\subsection{Key Findings from Recent Research}

Recent research highlights several crucial insights regarding gender bias and mitigation strategies in LLMs:

\textbf{Urchs et al. (2024)} examined ChatGPT’s responses in German and English. They observed that responses vary significantly depending on gendered prompts (male, female, neutral). German responses particularly showed grammatical challenges and an implicit default to masculine forms. This highlights the inadequacy of unstructured prompting methods for producing reliably neutral responses.

\textbf{Zeng et al. (2024)} focused on debias prompting across various models (GPT, Llama). Their experiments clearly demonstrated that structured prompting methods (zero-shot, few-shot, chain-of-thought) significantly reduce gender bias as measured by standard benchmarks like StereoSet and CrowS-Pairs. Few-shot examples were particularly effective, although they noted a performance-bias mitigation trade-off.

\textbf{Savoldi et al. (2024)} studied gender-neutral translation capabilities of GPT-4. Using systematic prompting approaches, they significantly improved GPT-4’s ability to generate gender-neutral translations (English → Italian). This study also pointed out the subjectivity involved in evaluating neutrality and acceptability, indicating the necessity of clear guidelines in educational content creation.

\textbf{You et al. (2024)} addressed biases related to the binary categorization of names by LLMs. They found low accuracy (below 40\%) for gender-neutral names, demonstrating the importance of recognizing and handling non-binary and gender-diverse identities. Their results underscore the necessity of inclusive representation, particularly important in diverse educational settings.

\textbf{Zhao et al. (2024)} analyzed gender bias across multiple languages, confirming that biases in LLMs extend beyond English. Their findings demonstrated biases in descriptive word choice, pronoun usage, and dialogue contexts. They also highlighted regional variations, emphasizing the importance of culturally and linguistically sensitive prompt design to ensure fairness in education.

\subsection{Synthesis and Educational Relevance}

Collectively, these findings underscore the significant impact that prompt engineering can have on mitigating gender biases in LLMs. Structured prompting techniques—such as clearly defined system messages, few-shot exemplars, and multi-step prompts—consistently produce more gender-neutral outputs. Furthermore, considering linguistic diversity and recognizing non-binary identities emerge as critical factors in developing inclusive educational content.

\subsection{Conclusion and Recommendation}

Structured prompting techniques, particularly few-shot and clearly defined system-message strategies, emerge as highly effective methods for ensuring gender neutrality in educational materials generated by LLMs. It is recommended that educators and content creators adopt these structured prompting practices to foster fair, inclusive, and equitable educational environments.


\textbf{Urchs et al. (2024)} examined ChatGPT’s responses … highlighting the inadequacy of unstructured prompting methods \cite{urchs2024chatgptbias}.

\textbf{Zeng et al. (2024)} demonstrated that structured prompting (zero‑shot, few‑shot, chain‑of‑thought) can significantly lower bias scores on benchmarks such as StereoSet and CrowS‑Pairs \cite{zeng2024debiasprompting}.

\textbf{Savoldi et al. (2024)} showed GPT‑4 can reach $\sim$70\% gender‑neutral translations with few‑shot prompts, far above baseline MT systems \cite{savoldi2024neutraltranslation}.

\textbf{You et al. (2024)} revealed that most LLMs exceed 80\% accuracy on binary name–gender prediction yet drop below 40\% for gender‑neutral names \cite{you2024beyondbinary}.

\textbf{Zhao et al. (2024)} confirmed gender bias across six languages and multiple discourse dimensions, underlining the importance of multilingual prompt design \cite{zhao2024multilangbias}.
