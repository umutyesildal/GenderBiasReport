\subsection{Gender Bias in Large Language Models}

Gender bias in language models manifests through systematic preferences for particular gender representations, occupational stereotypes, and linguistic patterns that reinforce societal inequalities \cite{zhao2024multilangbias}. These biases emerge from training data that reflect historical and contemporary gender disparities across professional, academic, and social domains.

In educational contexts, gender bias is particularly problematic as it can influence students' self-perception, career aspirations, and understanding of gender roles \cite{urchs2024chatgptbias}. Educational materials that consistently portray certain professions or characteristics as gender-specific can perpetuate stereotypes and limit students' potential for exploration across traditional gender boundaries.

\subsection{Prompting Strategies for Bias Mitigation}

Recent advances in prompt engineering have demonstrated significant potential for mitigating gender bias in LLM outputs. \textbf{Zeng et al. (2024)} conducted comprehensive experiments across multiple models (GPT, LLaMA) using structured prompting approaches including zero-shot instructions, few-shot examples, and chain-of-thought reasoning \cite{zeng2024debiasprompting}. Their findings indicate that few-shot prompting with explicit examples of gender-neutral language achieves substantial bias reduction on established benchmarks such as StereoSet and CrowS-Pairs.

\textbf{Savoldi et al. (2024)} specifically examined gender-neutral translation capabilities in GPT-4, achieving approximately 70\% gender-neutral translations through systematic prompting approaches—a significant improvement over baseline machine translation systems \cite{savoldi2024neutraltranslation}. Their work highlights the importance of explicit instruction and example provision in guiding models toward inclusive language generation.

\textbf{Urchs et al. (2024)} investigated ChatGPT's responses across German and English languages, revealing significant variations in gender bias depending on prompt formulation \cite{urchs2024chatgptbias}. Their research underscores the inadequacy of unstructured prompting methods for achieving consistent gender neutrality, particularly in morphologically rich languages where grammatical gender compounds bias effects.

\subsection{Evaluation Approaches and Metrics}

Evaluating gender bias in generated text requires sophisticated approaches that balance bias detection with assessment of text quality and semantic preservation. \textbf{You et al. (2024)} revealed critical limitations in current evaluation approaches, demonstrating that while LLMs achieve over 80\% accuracy in binary gender prediction tasks, performance drops below 40\% for gender-neutral names \cite{you2024beyondbinary}. This finding highlights the importance of comprehensive evaluation frameworks that account for non-binary and gender-diverse representations.

Current evaluation approaches typically employ combination of automated detection methods using regular expressions or trained classifiers, alongside human evaluation for fluency and acceptability \cite{savoldi2024neutraltranslation}. However, the subjectivity inherent in gender bias assessment necessitates clear evaluation criteria and multiple assessment dimensions.

\subsection{Gaps in Current Research}

Despite significant progress in understanding and mitigating gender bias in LLMs, several gaps remain:

\begin{enumerate}
    \item \textbf{Educational Domain Specificity}: Most existing research focuses on general text generation or specific linguistic tasks rather than educational content where pedagogical considerations are paramount.
    
    \item \textbf{Systematic Strategy Comparison}: While individual prompting approaches have been studied, comprehensive comparative analyses across multiple strategies using identical evaluation frameworks remain limited.
    
    \item \textbf{Trade-off Analysis}: Limited research examines the relationships between bias reduction, text fluency, and semantic preservation—critical considerations for practical implementation.
    
    \item \textbf{Verification Mechanisms}: The potential of self-verification and correction mechanisms in prompting strategies has not been thoroughly investigated.
\end{enumerate}

Our research addresses these gaps by providing a systematic comparison of prompting strategies specifically within educational content domains, using comprehensive evaluation metrics that account for both bias mitigation and text quality preservation.